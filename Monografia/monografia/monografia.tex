\documentclass[a4paper,12pt,titlepage]{article}
\usepackage[T1]{fontenc}
\usepackage[portuguese]{babel}
\usepackage[utf8x]{inputenc}
\usepackage{indentfirst}
\usepackage{graphicx}
\usepackage{times}
\usepackage{ucs}
\usepackage{float}    
\usepackage{fancyvrb}   
\usepackage{verbatim}
\usepackage{listings}
\usepackage{hyperref}

\hyphenation {pro-gra-ma pro-gra-mas com-pu-ta-cio-nais e-exem-plo pro-ble-mas
i-deia} 

\hypersetup{
    colorlinks=true,       % false: boxed links; true: colored links
    linkcolor=black,          % color of internal links (change box color with linkbordercolor)
    citecolor=black,        % color of links to bibliography
    filecolor=black,      % color of file links
    urlcolor=blue           % color of external links
}

\title{Trabalho de Conclusão de Curso \\
Animação de Fenômenos Físicos}
% ou ? Biblioteca Gráfica para Simulações de Física na Computação} 

\author{Rafael Issao Miyagawa \and Alberto Hideki Ueda \and 
		João Pedro Kerr Catunda \\ \ \and
        Orientador: José Coelho de Pina Junior }
\date{Novembro de 2012}

\begin{document}

\maketitle
\tableofcontents
\pagebreak

\part{Parte objetiva}

\section{Introdução} \label{introducao}

O foco deste projeto é atrair a atenção dos alunos do IME em relação a disciplina de física ministrada no curso de bacharelado em ciência da computação.
Com o simulador podemos integrar melhor os alunos aos assuntos abordados na física com demonstrações de ambientes físicos, 
integrando exercícios programas (EP) e também para ser utilizado em sala de aula.

\subsection{Física na computação}

A disciplina de Física (FAP-0126), oferecida no curso de BCC, é puramente teórica e não mostra nenhuma relação com a Ciência da Computação. Isso torna a disciplina menos interessante e frequentemente faz os alunos pensarem: "Para que serve esta disciplina?".
Para motivar os alunos e ilustrar melhor a relação entre as disciplinas básicas (Física, Estatística, Álgebra e Cálculo) com a Ciência da Computação, pretendemos criar uma biblioteca gráfica de simulação. Esta biblioteca será capaz de realizar uma leitura de dados de uma simulação de um EP e mostrar graficamente o resultado da simulação, por exemplo.
Esta biblioteca também proporcionará um ambiente de simulação específico e pronto para ser mostrado em salas de aula.
\subsection{Integração com a computação}

Existem muitos problemas ao tentar simular um ambiente físico com a computação. Temos o problema do tempo de simulação, 
que é o tempo que damos para os objetos físicos se interagirem e tomar o rumo necessário para refletir a realidade.
Existem alguns conceitos como broad phase, que é a fase em que os objetos são filtrados para realizarmos depois a narrow phase que é onde verificamos se aconteceu 
alguma colisão entre os objetos.

\subsection{Simulador}

A idéia do simulador é poder mostrar os problemas que encontramos ao tentar simular um ambiente físico com montagem de demonstrações. 

\newpage

\section{Plataforma computacional} \label{plataforma}
\subsection{Esquema de dependências }
\subsection{Ruby}

\subsection{Simulação com Chipmunk}

interface/envoltório (\textit{wrapper})

\subsection{Animação com Gosu}
\subsection{Cenários com Glaade}
\subsection{Exemplo}

\newpage

\section{Discretização da simulação} \label{discretizacao}

\subsection{Tempo de simulação}

\subsection{\textit{Time step problem}}
Descrevemos a dificuldade em criar interações físicas reais utilizando a computação.

\subsection{Problema de granularidade}
Qual deve ser o passo de simulação para ter uma boa velocidade de renderização e precisão física.

\newpage

\section{Colisões} \label{colisoes}
Uma das características das plataformas que suportam simulações físicas como Chipmunk é a capacidade de detectar colisões entre os objetos. 
Para detectar colisões de forma eficiente as plataformas utilizam um processo com duas fases: broad phase e a narrow phase. 

A finalidade da broad phase é evitar a realização de cálculos caros para corpos distantes umas das outras. O Chipmunk suporta as seguintes estruturas de dados para
gerar os pares de objetos que devem passar pelo algoritmo de detecção de colisão:
árvores AABB, 1D Sort and Sweep e o Spatial Hashing. Na versão 5 do Chipmunk a estrutuda de dados utilizado é o Spatial Hashing enquanto na versão 6
é o algoritmo árvores AABB. 

Narrow phase é a fase onde pares de objetos são verificados cuidadosamente por colisão. O Chipmunk suporta somente polígonos convexos nesta fase e devido a 
esse motivo, não é possível criar objetos côncavos.

Nas próximas seções serão explicadas resumidamente os algoritmos de cada fase.

\subsection{Broad Phase - Árvore AABB}

Para enteder a árvore AABB, é preciso primeiro entender o que é AABB. AABB é um acrônimo para Axis Aligned Bounding Box que são caixas delimitadoras de objetos 
em formas de retângulo e alinhado com os eixos x e y como mostra a figura 3:

\begin{figure}[!htbp]
  \centering
  \includegraphics[scale=0.3]{mario_bb.png}
  \caption{O retângulo azul representa uma caixa delimitadora do objeto}
\end{figure}

\ \\
A árvore AABB é uma árvore binária que faz proveito da AABB para armazenar e consultar objetos no espaço 2D. O nó da árvore pode guardar o próprio objeto 
ou possuir dois nós filhos. Quando um nó guarda um objeto, a caixa delimitadora deste nó deve conter a caixa delimitadora do objeto. Se o nó possui nós filhos, 
sua caixa delimitadora deve conter as caixas delimitadoras dos nós filhos. 
Um ponto importante é que esta estrutura precisa de uma heurística para inserir os nós na árvore. No caso do Chipmunk existe uma heurística baseada na caixa 
delimitadora do objeto e também em relação ao velocidade dela. Na figura 4 mostramos um exemplo de uma árvore AABB de três objetos A, B e C: 

\begin{figure}[!htbp]
  \includegraphics[scale=0.4]{AABBTree.png}
  \caption{Exemplo de uma árvore AABB}
\end{figure}

\ \\
A caixa delimitadora da raiz é o retângulo preto contendo os objetos A, B e C.
As duas novas caixas delimitadoras criadas pela linha indicado pelo número 1 são agora filhas da raiz. 
E finalmente a linha indicado pelo número 2 criam mais duas caixas delimitadoras separando os objetos A e B.

A figura 5 mostra como são definidos os pares de objetos que passarão pela narrow phase. Começando pela raíz, a caixa delimitadora do objeto D intersecta somente 
com a caixa delimitadora do nó esquerdo. Em seguida a caixa delimitadora do objeto D intersecta somente com a caixa delimitadora da folha contendo o objeto A.
Então o par criado nessa busca é o par \{A, D\}.

\begin{figure}[!htbp]
  \includegraphics[scale=0.4]{AABBTree1.png}
  \caption{Exemplo de detecção de colisão. O objeto D é comparado somente com a folha contendo o objeto A.}
\end{figure}

\subsection{Broad Phase - 1D Sweep and Prune}

A idéia do algoritmo é varrer as caixas delimitadoras dos objetos criando assim os pares de objetos que deverão ser passados para a fase seguinte. 

Seguindo o exemplo da figura 7, o algoritmo mantêm uma lista de objetos que estão sendo varridos. Quando a varredura encontra o início do objeto, ela o inclui na lista. 
Quando a varredura encontra o final do objeto, ela o exclui da lista. No exemplo observamos que a lista [A, B, C] criam os pares \{A, B\}, \{A, C\} e \{B, C\} 
enquanto a lista [B, D] cria o par \{B, D\}. 

\begin{figure}[!htbp]
  \includegraphics[scale=0.7]{sp.png}
  \caption{Exemplo de um algoritmo Sweep and Prune}
\end{figure}

Esse algoritmo, de acordo com a documentação do Chipmunk 6, Ele pode ser muito eficiente em jogos móveis se o seu mundo é muito comprido e plano 
como um jogo de corrida

\subsection{Broad Phase - Spatial Hashing}

Spatial Hashing é um processo onde o espaço de duas ou três dimensões é projetado em uma tabela hash de uma dimensão. 

\newpage

\section{Atividades realizadas} \label{atividades}
Nesta seção explicamos o processo da realização do trabalho, desde o desenho do projeto no papel e ideias iniciais até a criação dos ambientes físicos de demonstração, interface de criação de cenários (Physimulation) e finalmente as integrações com exercícios-programas.

\subsection{Estudo inicial}
Após nosso orientador nos apresentar as discussões e ideias do Grupo de Apoio ao BCC (seção \ref{introducao} e \ref{apendice}), decidimos planejar nosso trabalho de formatura levando em conta esta preocupação com os alunos do BCC. Inicialmente, nosso foco seria nas disciplinas de física (FAP-126) e estatística (MAE-121). \\

Partimos para a leitura de artigos e páginas da internet que tivessem relação com esta ideia. A seguir listamos algumas das referências mais interessantes neste processo:
\begin{itemize}
	\item \textit{An Introduction to Computer Simulation Methods: Applications to Physical Systems}, Gould, Tobochnik
	\item \textit{Evaluation of real-time physics simulation systems}, Boeing e Bräunl
	\item \textit{Esquema de detecção e resposta a colisões para animação física simplificada}, Temistocles, Atencio
	\item Box2D
	\item Hotruby
\end{itemize}

Percebemos que muitos estudos já havia sido feito na área de simulações e animações físicas, motivados principalmente pela criação de jogos para computador, \textit{video-games} e celulares. Neste último campo, com a recente expansão do uso de \textit{smartphones}, havia um número considerável de bibliotecas voltadas para reprodução de ambientes físicos. Durante esta fase pesquisa, o projeto começava a tomar mais forma e descartamos a possibilidade de trabalhar com a disciplina de estatística. A partir deste momento, concentraríamos esforços apenas em física. \\

Pesquisamos então as bibliotecas \textit{open-source} disponíveis e adequadas ao nosso objetivo de integração com a disciplina do IME.
Como já havíamos decidido trabalhar com a linguagem Ruby (seção \ref{plataforma}), procuramos por \textit{frameworks} nesta linguagem. Nesta fase de pesquisa, as bibliotecas que selecionamos para nosso trabalho foram o Chipmunk e Gosu (seção \ref{plataforma}). 

\subsection{Ambientes de demonstração}
Após definida a linguagem e as ferramentas que utilizaríamos no trabalho, partimos...

\begin{figure}[H]
	\centering
	\includegraphics[scale=0.2]{images/tcc-demos.jpg}
	\caption{Primeiros rascunhos das demonstrações}
\end{figure}

\subsection{Criador de Cenários}
Com um conhecimento mais profundo das bibliotecas após a implementação dos ambientes de demonstração, começamos a montar a interface de criação de cenários físicos. Como já tínhamos esta ideia em mente desde o início do projeto, centralizamos boa parte do código que era importante em classes que facilitariam a criação de simulações genéricas. Esta estratégia nos facilitou bastante o trabalho desta etapa.\\

Encontramos o glade...

\begin{figure}[H]
	\centering
	\includegraphics[scale=0.3]{images/glade.png}
	\caption{Gnome Glade}
\end{figure}

Discutímos durante as reuniões com o orientador e o João Kerr como seria a interface, as propriedades físicas que poderiam ou não poderiam ser alteradas...

Falar da questão do momento de inércia...

\begin{figure}[H]
	\centering
	\includegraphics[scale=0.5]{images/physimulation.png}
	\caption{Interface ao final das discussões}
\end{figure}


\newpage

\section{Animações produzidas} \label{animacoes}
Apresentação e explicação dos demos.

\newpage

\section{Physimulation} \label{physimulation}
Descrição da interface de criação de cenários e alguns exemplos do que pode ser criado com ele.


\begin{figure}[H]
	\centering
	\caption{Cenário 1}
	\includegraphics[scale=0.3]{images/cenario-two-balls.png}
	\hspace{0.5cm}
\end{figure}  

  \begin{figure}[H]
	  \centering
	  \caption{Cenario 2}
    \includegraphics[scale=0.4]{images/cenario-todos.png}
  \end{figure}

  \begin{figure}[H]
	  \centering
	  \caption{Cenario 2 - Raio-X}
    \includegraphics[scale=0.4]{images/cenario-todosE.png}
  \end{figure}

  \begin{figure}[H]
	\centering
	\caption{Cenário Gravitacão 1}
    \includegraphics[scale=0.4]{images/cenario-gravitacao-2.png}
    \includegraphics[scale=0.4]{images/cenario-gravitacao.png}
  \end{figure}

  \begin{figure}[H]
	\centering
	\caption{Cenário Gravitacão 2}
    \includegraphics[scale=0.4]{images/cenario-gravitacao-4.png}
    \includegraphics[scale=0.4]{images/cenario-gravitacao-3.png}
  \end{figure}


\newpage

\section{Introdução à computação com animações} \label{ep}
Nesta seção explicamos as duas integrações que fizemos com exercícios-programas dados em disciplinas de introdução a computação e que envolviam física.

\subsection{Configuração}


os enunciados estão


\subsection{Angry Bixos}

\begin{itemize}
  \item yE , vmax : 2 reais que definem a posição do estilingue e a velocidade máxima com que um bixo pode ser arremessado;
  \item yA , hA : 2 reais que definem a posição do alvo e sua altura;
  \item dist: real positivo que define a distância xA − xE entre o alvo e o estilingue.
  \item nBix: inteiro que define o número de bixos que podem ser lançados;
  \item nLin, nCol: 2 inteiros que definem as dimensões do gráfico a ser impresso (ambos devem ser múltiplos de 5);
  \item nUni: real utilizado como fator de escala das alturas (número de unidades de altura por linha);
  \item g: real negativo que define a aceleração da gravidade.
\end{itemize}

\begin{figure}[H]
    \centering
	\includegraphics[scale=0.6]{images/angry-bixos-3.png}
	\includegraphics[scale=0.22]{images/angry-bixos-4.png}
	\includegraphics[scale=0.22]{images/angry-bixos-4E.png}
	\caption{Integração com EP Angry Bixos}
\end{figure}

\subsection{Apolo}

\begin{itemize}
  \item z1) posição inicial de uma nave em coordenadas cartesianas;                    
  \item z2) as componentes (V\_X,V\_Y) do vetor velocidade inicial da nave;
  \item z3) tempo máximo de simulação;               
  \item z4) o intervalo dT entre um instante e o instante seguinte da simulação.
\end{itemize}

\begin{figure}[H]
    \centering
	\includegraphics[scale=0.22]{images/apolo-4.png}
	\includegraphics[scale=0.22]{images/apolo-6.png}
	\includegraphics[scale=0.22]{images/apolo-3.png}
	\includegraphics[scale=0.22]{images/apolo-2.png}
	\includegraphics[scale=0.22]{images/apolo-1.png}
	\includegraphics[scale=0.22]{images/apolo-5.png}
	\caption{Efeito \textit{slingshot} Integração com EP Apolo}
\end{figure}



\newpage

\section{Comentários finais} \label{comentarios}

Com o desenvolvimento do Physimulation e as integrações com exercícios-programas passamos por um grande aprendizado nas áreas de simulação e animação, programação - principalmente por utilizarmos Ruby, uma linguagem nova para nós - e física. Nossa intenção é que boa parte deste aprendizado seja repassado para os alunos de física e computação em geral, assim como a motivação para o estudo de disciplinas teóricas do IME. \\

Para os alunos interessados em discretização de simulações, seja para trabalhos acadêmicos ou para jogos que envolvam algum tipo de simulação física, recomendamos a leitura da seção \ref{discretizacao}, em que mostramos um pouco do que aprendemos em relação a tempo de simulação. Dependendo do objetivo do aluno, este tempo deverá ser fixo ou variável. Além disso, há a questão da simulação ser ou não em tempo real: há casos em que é mais importante ter uma animação com resultados precisos e realistas do que uma resposta imediata, mesmo que o processamento demande horas de cálculos. \\

O estudo de alguns conceitos e algoritmos de geometria computacional para detecção e tratamento de colisões também foi bastante interessante. A seção \ref{colisoes} possui referências para o aluno que desejar um conhecimento mais profundo desta área, que é inclusive uma das disciplinas eletivas oferecidas pelo IME atualmente. (TODO confirmar) \\

Como visto na seção \ref{physimulation}, o código do Physimulation foi construído com uma preocupação constante: ser fácil de entender e de modificar. Esperamos que com isso os alunos se sintam mais à vontade para utilizar nosso trabalho ou mesmo alterá-lo, dando continuidade ao projeto. Alguns trabalhos futuros estão descritos na parte subjetiva desta monografia. \\

Assim, esperamos que o Physimulation seja de fato uma contribuição, mesmo que pequena, à questão de contextualização das disciplinas do IME. Como foi dito na apresentação do trabalho, nossa expectativa é que esta iniciativa dê resultados a médio e longo prazo, tendo como horizonte a utilização do Physimulation em salas de aula pelo próprio professor e em conjunto com exercícios teóricos e EP's.

\newpage

%\bibliographystyle{amsalpha}
%\bibliography{bibliografia}
\newpage

\section{Roteiro de instalação da plataforma} \label{instalacao}
Passos para instalação das bibliotecas: ruby, chipmunk, gosu, chingu, etc.

Falar do warning do próprio chipmunk:
gems/ruby-1.9.3-p286/gems/chipmunk-5.3.4.5/lib/chipmunk.rb:6: Use RbConfig instead of obsolete and deprecated Config

(Talvez seja um apêndice)

\newpage

\section{Apêndice} \label{apendice}
O texto a seguir foi retirado do site: \url{http://bcc.ime.usp.br/principal/index.php?id=material-de-apoio}. \\

{\large \textbf{Material de Apoio ao BCC}} \\

A falta de contextualização das disciplinas básicas do BCC tem sido uma queixa recorrente do alunos nas reuniões entre alunos e professores, no Encontro do BCC de 2010 e também no processo de avaliação semestral que é realizado pelo orientador pedagógico orientador pedagógico da Escola Politécnica (POLI), Giuliano Salcas Olguin.

A fim de motivar os alunos e ilustrar a relação entre ciência da computação e as disciplinas básicas de álgebra, cálculo, estatística, probabilidade e física presentes no currículo do BCC a CoC sugeriu que fossem produzidos documentos ilustrando aplicação de cada uma dessas disciplinas em ciência da computação e vice-versa.  Esses documentos têm o objetivo de motivar os alunos do BCC:

\begin{enumerate}
\item ilustrando as relações entre as disciplinas básicas do curso e ciência da computação;
\item mostrando aos alunos quais das disciplinas mais avançadas do BCC que fazem uso dos conteúdos das disciplinas básicas;
\item fornecendo aos professores das disciplinas básicas do BCC exemplos de aplicações de suas especialidades em ciência da computação, que, eventualmente, podem ser mencionados em aulas ou ser temas de trabalhos.
\end{enumerate}

Esses documentos poderão também ser usados pelas disciplinas de Introdução à Ciência da Computação que são oferecidas pelo DCC para várias unidades da USP. Nestas disciplinas, frequentemente, os chamados exercícios programas ilustram aplicações de métodos computacionais na solução de problemas em genômica, física, economia, etc.
Por exemplo, na última edição da disciplina \begin{quote} MAC2166 Introdução à Ciência da Computação para Engenharia \end{quote}
podemos ver um exercício programa em que é simulada a "trajetória de livre de retorno" de uma nave sob a ação gravitacional da Terra e da Lua em \url{http://www.ime.usp.br/~mac2166/ep3/}. Já um exercício programa com aplicação em genômica pode ser visto em \url{http://www.ime.usp.br/~mac2166/ep4/}.

Além de uma maior integração do curso este projeto pretende propor possíveis mudanças na grade curricular do BCC. Para isto pretendemos realizar uma pesquisa com o egresso do BCC e uma pesquisa das grades curriculares dos cursos de computação pelo mundo.

\newpage

\part{Parte subjetiva}
Nesta seção descreveremos a relação entre nosso projeto e a experiência adquirida no BCC.

(TODO pensei em fazer separado mas agora que escrevi tenho impressão que não vai mudar muita coisa)

\section{Alberto Ueda}
Entregar este projeto como trabalho de formatura e disponibilizar seu código para os alunos do BCC foram duas das experiências mais gratificantes que já tive. Isto pois acredito que tal conteúdo poderá ser utilizado pelas próximas turmas do BCC como incentivo ao aprendizado da matéria de física. Além disso, tanto alunos do próprio Instituto de Física quanto da Engenharia Politécnica também poderão se interessar pelo conteúdo: o primeiro grupo (FIS) pela animação de fenônemos físicos estudados e o segundo (Poli) tanto pela animação quanto pela simulação de tais fenônemos.

Mas, ao mesmo tempo, por ser um trabalho que levou meses, certas dificuldades foram encontradas pelo caminho. Tivemos que tomar decisões às vezes frustrantes, porém necessárias.

\subsection{Desafios e frustrações encontrados}
Inicialmente, nossa motivação era entregar um sistema que utilize recursos do Wii Remote (TODO ref TODO link da caneta) e que o professor pudesse utilizá-lo em sala de aula para realizar suas simulações e animações. Porém, chegamos a conclusão que esta tecnologia aumentaria consideravelmente o nível de complexidade de nosso trabalho e não tínhamos garantia de que utilizá-la acrescentaria da mesma forma ao resultado final. Assim descartamos esta possibilidade.

Como utilizamos algumas bibliotecas de terceiros em nosso projeto, tivemos que entender obrigatoriamente como eram feitas as principais chamadas de métodos destas bibliotecas, principalmente o Chipmunk e o Gosu. Um detalhe interessante que ocorreu no segundo mês de trabalho foi a necessidade de mudar o código da biblioteca (TOOD citar) e recompilá-la para que uma função simples de mensagem para o usuário funcionasse (TODO conferir método). Uma semana depois, utilizando uma versão mais nova da biblioteca, descobrimos que nossa alteração não era mais necessária, pois já havia sido feita pelos próprios programadores na mudança de versão.

Além disso, utilizamos um bind (TODO envoltório?) da versão original do Chipmunk. Isto trazia duas dificuldades para nós: 1) o código original (em C++) sempre estava com uma versão mais recente e provia (TODO conferir) mais métodos; e 2) nem sempre o que víamos na documentação oficial possuia correspondente em nosso bind. 
 
Por último, um desafio que tivemos foi encontrar um professor de física disponível para nos auxiliar na elaboração do protótipo do sistema. Ficamos muito felizes quando após algumas semanas o bacharel em física e aluno do BCC João Kerr veio a uma de nossas reuniões, a convite do professor Coelho.

\subsection{Disciplinas mais relevantes}

(TODO comentar)
\begin{itemize}
\item MAC0110 	Introdução à Computação 
\item FAP0126 	Física I 
\item MAC0122 	Princípios de Desenvolvimento de Algoritmos 
\item MAC0211 	Laboratório de Programação I 
\item MAC0323 	Estruturas de Dados 
\item MAC0420 	Introdução a Computação Gráfica 
\item MAT0211 	Cálculo Diferencial e Integral III 
\item MAC0242 	Laboratório de Programação II 
\item MAC0316 	Conceitos Fundamentais de Linguagens de Programação 	
\item MAC0332 	Engenharia de Software 
\item MAC0338 	Análise de Algoritmos
\item MAC0446 	Princípios de Interação Homem-computador 
\item FAP0137 	Física II 
\end{itemize}
\subsection{Estudos futuros} 
Sem dúvida os tópicos de estudo mais importantes para a continuação deste trabalho são as disciplinas de Física I e II para o BCC. Quanto maior o conhecimento das leis e forças físicas presentes no mundo real, melhor serão as simulações e consequentemente as animações geradas.

Em segundo lugar, seria interessante uma análise de qual das alternativas a seguir tem uma melhor relação custo-benefício, visando a atualização do projeto com a versão mais nova do Chipmunk: A) migrar nosso projeto de Ruby para C++ e usar diretamente a versão original do Chipunk, sem binds; ou B) atualizar o bind em Ruby adicionando os métodos e funcionalidades da versão mais recente em C++.

Por último, mas não menos importante, um estudo de paradigmas que proporcionem mais usabilidade ao usuário, substituindo o preenchimento obrigatório de formulários para criação de objetos físicos. Ex: drag-and-drop do mouse para "arrastar" as formas geométricas, fornecendo os valores de massa, coeficientes de elasticidade e atrito a posteriori (após o objeto já estar na tela).


\section{Rafael Miyagawa}

\newpage 

\end{document}

