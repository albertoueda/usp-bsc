A simulação no Chipmunk ocorre em intervalos de tempo, ou seja, precisamos indicar o tempo que os objetos vão interagir como vimos na seção 2. 
Esse intervalo que passamos para o Chipmunk é o tempo de simulação. É preciso definir bem esse valor para que a simulação não fique muito rápido 
ou muito lento. A escolha desse tempo parece trivial mas existem muitas implementações que indicam como definir esse tempo de simulação. Nas 
próximas seções apresentaremos as implementações para tempo de simulação fixo e variável.

\subsection{Fixo}

A maneira simples é fixar o tempo de simulação, como por exemplo 1/60 de um segundo:

\begin{lstlisting}[language=Ruby, caption=Implementação de tempo de simulação fixo]
def @dt = 1/60

while (running) do
  $space.step(@dt)
end
\end{lstlisting}

Na maioria das simulações o código acima é o ideal. Se sua simulação coincidir com a taxa de atualização da animação (frame rate) 
e ainda se o método \$space.step(@dt) não demorar mais do que 1/60 de um segundo, então a solução é perfeita. Mas no mundo real não sabemos
o valor do display refresh rate. Por exemplo, a simulação em uma máquina lenta, não pode atualizar o frame rate em 60fps e tornar o seu 
quadro de rápido o suficiente para apresentá-lo a 60fps. Nesse caso a simulação ficaria mais lenta. Para evitar estas situações podemos 
implementar o tempo de simulação variável.

\subsection{Variável}

Escrever
